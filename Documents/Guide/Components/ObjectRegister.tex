\subsection{ObjectRegister}
\label{ObjectRegister}

The objectRegister is a singleton object [it should be per
  simulation], which keeps each and then deletes when at its lifetime
end, each object registered with it. It only accepts two types of
object, a dummy name object and a FixedComp object. 

If a dummy object is required, the name (and possibly number) of the object 
is provided and the objectRegister singleton provides a unique
range of cell and surface numbers, typically 10,000 units of each, but
can be user selected. This is its only responsibility and to ensure that the
name is unique. 

Significantly more complex is the FixedComp registration, in this case
a \prog{std::shared\_ptr} of FixedComp must be provided by the
calling method. Obviously, for a shared\_ptr the object memory must be
allocated, i.e. an initial \verb|new object(...)| is normally called
directly or previously. A typical structrue might be:

\begin{verbatim}
std::shared_ptr<BeamPipe> A  = new BeamPipe("LongPipe");

ModelSupport::objectRegister& OR=
    ModelSupport::objectRegister::Instance();

OR.addObject(A);
\end{verbatim}

From this example, the BeamPipe class is inherrited from FixedComp,
this is manditory. A tempory reference \prog{OR} is created by calling
the static Instance() method. All singletons in CombLayer provide an
Instance() method for this purpose. Then the object pointer is referenced
to the objectRegiste with \prog{addObject}.

However, hidden from view is a call to objectRegister in FixedComp's
constructor, which is certain to be called as all registered object must derive
from this class. That occures during the operator new call and results
in the allocation of the cell/surface numerical range. If it is
necessary to trap that error, the try/catch block must be around the
new opertor. The main exception that is possible if an existing object
already exists with the same name.



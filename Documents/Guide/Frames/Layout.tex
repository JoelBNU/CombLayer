\section{Layout}
\label{Sec:Layout}

The basic program structure is given by figure \ref{MainLayout}.  The
main program structure is normally copied from an existing project and
the areas are constructed by the user. It is normal flow is to to call
functions that: (i) define new input options to enter parameters from the
command line, (ii) variables that your project is going to use,
(iii)  build the geometry via a call to a makeProject functoin (iv)
set up tallies (v) generate variance reduction.There are other ways
to construct the system but this allows a degree of autonomy from
tallies/variance reduction and producing an appropiate output.

\begin{figure}[!htp]
\begin{tikzpicture}

  \coordinate (Main) at (0,0);
  \coordinate (Input) at (3,-2);
  \coordinate (Var) at (3,-4);
  \coordinate (Project) at (3,-6);
  \coordinate (Object) at (6,-8);
  \coordinate (Tally) at (3,-10);
  \coordinate (Variance) at (3,-12);
  \coordinate (Output) at (3,-14);

  \tikzstyle{fbox}=[draw=blue,very thick,
                    shape=rectangle,rounded corners=0.5em,
                    inner sep=4pt,
                    minimum height=2em,text badly centered,
                    fill=orange!90!white];
  \tikzstyle{obox}=[draw=blue,very thick,
                    shape=rectangle,rounded corners=0.5em,
                    inner sep=4pt,
                    minimum height=2em,text badly centered,
                    fill=blue!10];

%  \draw[blue,line width=0.4mm,->] (tally) (tally) -- (weight);
%  \draw[blue,line width=0.4mm,->] (tally) (geom) -- (weight);
%  \draw[blue,line width=0.4mm,->] (tally) (var) -- (geom);
%  \draw[blue,line width=0.4mm,dashed,->] (tally) -- (geom);


  \node[fbox] (MainBox) at (Main) { \bf \large Main Program } ;

  \node[fbox] (InputBox) at (Input) { \bf \large ProjectInputs };
  \draw[->,very thick,color=red,>=triangle 60] (MainBox) |- 
                   node[left]{} (InputBox) ;

  \node[fbox] (VarBox) at (Var) { \bf \large ProjectVariables };
  \draw[->,very thick,color=red,>=triangle 60] (MainBox) |- 
                   node[left]{} (VarBox) ;

  \node[fbox] (ProjectBox) at (Project) { \bf \large makeProject };
  \draw[->,very thick,color=red,>=triangle 60] (MainBox) |- 
                   node[left]{} (ProjectBox) ;

  \node[obox] (ObjectBox) at (Object) { \parbox{3.0cm}{ \bf \large Object A\\
 Object B\\ Object C} };
  \draw[->,very thick,color=red,>=triangle 60] (ProjectBox) |- 
                   node[left]{} (ObjectBox) ;

  \node[obox] (TallyBox) at (Tally) { \bf \large Tallies };
  \draw[->,very thick,color=red,>=triangle 60] (MainBox) |- 
                   node[left]{} (TallyBox) ;

  \node[obox] (VarianceBox) at (Variance) { \bf \large Variance Reduction };
  \draw[->,very thick,color=red,>=triangle 60] (MainBox) |- 
                   node[left]{} (VarianceBox) ;

  \node[obox] (OutputBox) at (Output) { \bf \large Output };
  \draw[->,very thick,color=red,>=triangle 60] (MainBox) |- 
                   node[left]{} (OutputBox) ;
\end{tikzpicture}
\caption{The main program calling sequence is shown. The parts
  in orange, are expected to be constructed by the user. Bespoke objects
  can be added for a project but it is not necessary.  }
\label{MainLayout}
\end{figure}


\subsection{Main}

The main function for CombLayer follows a relatively linear template.
Consider the example

\begin{lstlisting}[language=C++]{Name=MainProg}{floatplacement=H}

int 
main(int argc,char* argv[])
{
  int exitFlag(0);                // Value on exit
  ELog::RegMethod RControl("","main");
  mainSystem::activateLogging(RControl);
  std::string Oname;
  std::vector<std::string> Names;  
  std::map<std::string,std::string> Values;  

  // PROCESS INPUT:
  InputControl::mainVector(argc,argv,Names);
  mainSystem::inputParam IParam;
  %%{\tcgg{\bf createPipeInputs(IParam);}@

  Simulation* SimPtr=createSimulation(IParam,Names,Oname);
  if (!SimPtr) return -1;

  // The big variable setting
  %%{\tcgg{\bf setVariable::PipeVariables(SimPtr-$>$getDataBase());} @
  InputModifications(SimPtr,IParam,Names);
  
  // Definitions section 
  int MCIndex(0);
  const int multi=IParam.getValue<int>("multi");
  try
    {
      SimPtr->resetAll();
      
     %%{\tcgg{\bf pipeSystem::makePipe pipeObj;}@

      World::createOuterObjects(*SimPtr);
     %%{\tcgg{\bf pipeObj.build(SimPtr,IParam)}; @  
      SDef::sourceSelection(*SimPtr,IParam);
      
      SimPtr->removeComplements();
      SimPtr->removeDeadSurfaces(0);         
      ModelSupport::setDefaultPhysics(*SimPtr,IParam);
      
      const int renumCellWork=tallySelection(*SimPtr,IParam);
      SimPtr->masterRotation();
      if (createVTK(IParam,SimPtr,Oname))
	{
	  delete SimPtr;
	  ModelSupport::objectRegister::Instance().reset();
	  ModelSupport::surfIndex::Instance().reset();
	  return 0;
	}
      
      if (IParam.flag("endf"))
        SimPtr->setENDF7();
      
      SimProcess::importanceSim(*SimPtr,IParam);
      SimProcess::inputPatternSim(*SimPtr,IParam); // energy cut etc
      
      if (renumCellWork)
	tallyRenumberWork(*SimPtr,IParam);
      tallyModification(*SimPtr,IParam);
      
      if (IParam.flag("cinder"))
	SimPtr->setForCinder();
            
      // Ensure we done loop
      do
	{
	  SimProcess::writeIndexSim(*SimPtr,Oname,MCIndex);
	  MCIndex++;
	}
      while(MCIndex<multi);
      
      exitFlag=SimProcess::processExitChecks(*SimPtr,IParam);
      ModelSupport::calcVolumes(SimPtr,IParam);
      ModelSupport::objectRegister::Instance().write("ObjectRegister.txt");
    }
  catch (ColErr::ExitAbort& EA)
    {
      if (!EA.pathFlag())
	ELog::EM<<"Exiting from "<<EA.what()<<ELog::endCrit;
      exitFlag=-2;
    }
  catch (ColErr::ExBase& A)
    {
      ELog::EM<<"EXCEPTION FAILURE :: "
	      <<A.what()<<ELog::endCrit;
      exitFlag= -1;
    }
  %%{\it delete SimPtr;} @
  ModelSupport::objectRegister::Instance().reset();
  ModelSupport::surfIndex::Instance().reset();

  return exitFlag;
}
\label{MainProg}
\end{lstlisting}

The Main program given in listing \ref{MainProg} highlights the areas
that the user should be creating. The remainder of the main() function
deals with trapping exceptions, loggin and building variance reduction
and tallies into the model.

\begin{enumerate}[(i)]
\item {\bf createPipeInputs} is a 
function to define which command line options [above the standard ones]
this model should support. It doesn't do anything with them, just a
list of options, number of arguements they can take and any default
values that the options should take. All options defined here are
access from the command line option with a {\it -} sign. E.g. {\it -r} as a
renumber operation. In this form of the program, if the main program
is run without any options, a list and very brief description of each
option is shown (e.g. execute {\it ./pipe}).

If no additional options are required, a call to
{\it createInputs(IParam)} would be expected. Significant restructuring
would need to take place to avoid that call.
\item \tcgg{setVariable::PipeVariables} is the method that registers
and sets a default value for all the variables that the model will
use. 
\item \tcgg{makePipe pipeObj} and \tcgg{pipeObj.buid(SimPtr,IParam)} are the
main geometry building calls. Typically 100\% of the geometry is built
in this zone. It is not a place for tallies, variance reduction and other
non-geometry items.
\end{enumerate}




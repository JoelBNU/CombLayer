
\subsection{EXT Command}

MCNP(X) provides the EXT card for biasing the direction of the
particles after collisions. The card can be configured with a
stretching parameter value between -1.0 and 1.0 and an optional vector
or direction associated with it. If a vector is not given the
stretching parameter is applied in the direction of the neutron travel.

MCNP(X) only accepts the direction to be X,Y,Z which is highly
limiting in the CombLayer environment, so it is only partially supported.

The other two options vector and non-vector are supported.

\subsection{-wExt entry}

The first method of entry is via the command line option -wExt. This
command takes a sequence of additional values which are split into a
{\it zone} and {\it type} region. The {\it zone} region is based on
the cells that are to be biased. This can be give with the commands:

\begin{itemize}
\item{ {\bf all} : Apply to all non-void cells.}
\item{ {\bf Object [name]} : Apply to all objects within the object name.}
\item{ {\bf Cells [N1 N2...]} : Apply to given cell numbers [pre-renumber].}
\item{ {\bf Range [N1  N2]} : Apply to all cell numbers between N1 and N2.}
\end{itemize}

The {\it type} section depends on the type of EXT command required. The options are

\begin{itemize}
\item{ {\bf simple} : Simple $\Sigma_C/\Sigma_T$ scaling.}
\item{ {\bf simpleVec [Vec]} : Simple $\Sigma_C/\Sigma_T$ scaling.} 
\item{ {\bf Object [name]} : Apply to all objects within the object name.}
\item{ {\bf Cells [N1 N2...]} : Apply to given cell numbers [pre-renumber].}
\item{ {\bf Range [N1  N2]} : Apply to all cell numbers between N1 and N2. } 
\end{itemize}



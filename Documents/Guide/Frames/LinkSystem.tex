CombLayers geometry is composed of a set of objects that have slightly
stronger rules than a typical MCNPX model. Obviously any MCNPX model
can be represented as a CombLayer model and in the extreme case that
is done by defining one object to contain the MCNPX model. However, the
little benefit would be derived from such an approach.

The basic geometric system is to build a number of geometric classes
and construct the model by incorporating those into the desired
configuration. Each geometric class is designed to be built and an
arbitary position and rotation, be of an undetermined number, and
interact with its surroundings in a well defined manor. 

In object orientated programming, functional rules and properties are
normally added to objects by inherritance. CombLayer follows that
pattern. As such most geometry item classes inherit from base classes within 
the attachSystem namespace.

\subsection{AttachSystem Namespace}
\label{AttachSystem}

The CombLayer system is built around the interaction of FixedComp
units, ContainedComp units and LinkUnits. The use of these and their
interactions are the basic geometric building tools. These object
reside within the attachSystem namespace.

Almost any geometric item can be designated as a FixedComp
object. This is done by public inherriting from directly from the
FixedComp, or by inherriting from one of the more specialised attachSystem
objects e.g. TwinComp or LinearComp. 

\subsection{FixedComp}

The basic FixedComp object holds the origin and the orthoganal basis
set (X/Y/Z) for the geometry item being built. In addition it holds a 
number of LinkUnits which provide information about the outer (and/or inner)
surfaces and positions on the geometric item. 

As with all Object-Orientated (OO) constructions their is an implicit
contract that the inherited object should adhere to. This is normally
expressed as the {\it Liskov Substitution principle}: This principle
states that functions that use pointers/references to the base object
must be able to use the objects of derived classes without knowing
it. In this case, that means that modification of the origin or the
basis set should not invalidate it and that the object should do the
expected thing. E.g. if the origin is shifted by 10 cm in the X
direction the object should move by 10cm in the X direction. It also
means that the basis set must remain orthogonal at all time.

Other than providing an origin and an basis set, the FixedComp has a
number of link points. The link points are there to define joining
surfaces, points and directions. Each link point defines all three parts.

For example a cube might have 6 linkUnits, and each linkUnit would
have a point at the centre of a face, a direction that is normal to
the face pointing outwards and a surface definition that is the
surface pointing outwards. \[ Note that in the case that the link
points define an inner volume, for example in a vacuum vessel, then
the surfaces/normals should point towards the centre.\]

The actual link surface does not need to be a simple surface. In the
case, that an external surface needs multiple surfaces to define the
external contact these can be entered into a link-rule. For example,
if the cube above was replaces with a box with two cylindrical
surfaces the link surface would be defined as the out going cylinder
intersection with a plane choosing the side. 

In the case of an equiry for the linkSurface (e.g. to do an line
intersection) then it is the first surface that takes
presidence. However, all actions can be carried out on the link-rule
including line intersections etc.


\subsection{ContainedComp}

The ContainedComp defined both the external and interal enclosed
volume of the geometric item. It is most often used to exclude the
item from a larger enclosing geometric object: e.g. A moderator will
be excluded from a reflector, or it can be used to exclude a part of
the geometric item from another geometric object. E.g. two pipes which
overlap can have one exclude itself from the other. 

In CombLayer, the ContainedComp are considered the primary geometric
item, i.e. it is the ContainedComp that is removed from the other
items.  However, it is used in a two stage process wereby cells are
registered to be updated by the ContainedComp at a later date. This
was to allow forward dependency planning but has more or less been
superseded by the attachControl system.

\section{Installation}

CombLayer is predominately written for the Linux platform using C++
compilers that support C++11 or greater. The code is available from
https://github.com/SAnsell/CombLayer, either as a download of a zip file or
by cloning/pulling the git repository.

\subsection{Requirments}

CombLayer needs to have the Gnu Scietific Library [GSL] and the
boost::regex system along with the STL libraries from your c++
compiler. The GSL can be avoided with the NS flag in the make script
but some functionality will be lost, particularly in the choice
of variance reduction methods.

Additionally, the primary build system uses cmake. There is another
that just uses make but is significantly more time-consuming.

Functional documentation is supported using Doxygen and the construction
of new cmake text files can be done via PERL scripts.

Currenly it is know that gcc version 4.6 and above can compile
CombLayer as can clang (all tested versions). gcc 4.4 which is often
the default on RedHat systems (2015) does not work.

\subsection{Basic build method}

If a clean directory is made and then the .zip file is uncompressed, the
following commands should build a version of CombLayer.

\begin{verbatim}
./CMake.pl
cmake ./
make
\end{verbatim}

This should make a number of executables, e.g. ess / simple etc. These
can be used to make a simple model with commands like {\it ./simple -r
AA}. This will produce ans output file AA1.x which is an MCNP model.




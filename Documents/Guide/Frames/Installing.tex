\section{Installation}

CombLayer is predominately written for the Linux platform using {\tt C++}
compilers that support {\tt C++11} or greater. The code is available from \\
\href{https://github.com/SAnsell/CombLayer}{https://github.com/SAnsell/CombLayer}, \\ either as a download of a {\tt zip} file or
by cloning/pulling the git repository.

\subsection{Requirments}

CombLayer needs to have the GNU Scietific Library [GSL] and the {\tt
  boost::regex} system along with the STL libraries from your {\tt
  C++} compiler. The GSL can be avoided with the {\tt -NS} flag in the
{\tt getMk.pl} and the {\tt CMake.pl} script but some functionality will be
lost, particularly in the choice of variance reduction methods.

Additionally, the primary build system uses {\tt cmake}. There is another
that just uses {\tt make} but is significantly more time-consuming.

Functional documentation is supported using {\tt Doxygen} and the construction
of new cmake text files can be done via PERL scripts.

Currenly it is know that {\tt gcc} version 4.6 and above can compile
CombLayer as can {\tt clang} (all tested versions). {\tt gcc} 4.4 which is often
the default on RedHat systems (2015) does not work.

\subsection{Basic build method}

If a clean directory is made and then the {\tt .zip} file is uncompressed, the
following commands should build a version of CombLayer.

\begin{bash}
  ./CMake.pl
  cmake ./
  make
\end{bash}

This should make a number of executables, e.g. {\tt ess}, {\tt simple}, {\tt fullBuild} etc. These
can be used to make a simple model with commands like
\begin{bash}
  ./simple -r AA
\end{bash}
This will produce an output file {\tt AA1.x} which is a MCNP model.




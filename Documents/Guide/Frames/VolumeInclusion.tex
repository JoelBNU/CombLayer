%\section{HowTo}
\begin{itemize}
\item Get real surface number by its relative number: SMap.realSurf(divIndex+103) (see createLinks methods)
\end{itemize}

\subsection{How to put one object into another}

Suppose, we are inserting Spoon into Mug.
Mug is made up of N cells. Spoon is made of one contained component with outer surface.
CombLayer provides several methods to put one object into another:

\begin{cpp}{Name=essBuild}{floatplacement=H}
attachSystem::addToInsertForced(System,   *Mug, *Spoon);
attachSystem::addToInsertSurfCtrl(System, *Mug, *Spoon);
attachSystem::addToInsertControl(System,  *Mug, *Spoon);
attachSystem::addToInsertLineCtrl(System, *Mug, *Spoon);
\end{cpp}

\subsubsection{addToInsertForced}

The outer surface of the Spoon is excluded from the HeadRule of every single cell of Mug.
Even if Mug contains cells which do not intersect with Spoon (e.g. its handle).
{\it Forced} means {\it do it and do not think about it}, but at the same time it means that
{\it I have got something wrong somewhere.}  Normally this is that insufficient link points have been added
to the object, or that the object is a set of split (single cell) volumes.  However, there is the additional
problem that the model may not be correctly constructed at this point, so that the other options seem not to work.
This can be checked by adding a
\vb{SimProcess::writeIndexsim(System,\"OutputFilename.txt\",0);}
in the code just before


\subsubsection{addToInsertSurfCtrl}
First, it deconvolves Mug into its surfaces.
Then for each cell of Mug it calculates intersections between all surfaces of this cell and all surfaces of Mug.
The Mug is inserted only into those cells of Mug which it intersects.

It is not always better to call {\tt addToInsertSurfCtrl} instead of {\tt addToInsertForced}. \alert{Example needed.}

{\tt addToInsertSurfCtrl} is a very expensive function to call, because you have to check all the surface triplets. So, it runs a bit slower than addToInsertForced, but the geometry will be faster.
However, there is another method which provides the same trick for less pain.

\subsubsection{addToInsertControl}
It's a very simple method. Spoon has to have the link points defined.
The method checks if any of these link point fit inside the outer surface of Mug. If it does, then it cuts Spoon from the Mug.
It is possible to add a vector of link points to check as a parameter.

\subsubsection{addToInsertLineCtrl}
Imagine we have a (big) contained component~(Mug) and some (small) object which clips it~(Spoon). The link points are {\bf not} in the Mug~(therefore {\bf addToInsertControl} can not be used), but the lines which connect them are in the Mug.
The method checks the lines connecting the link points and sorts out the intersections.
